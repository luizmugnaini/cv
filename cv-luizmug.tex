%%%%%%%%%%%%%%%%%%%%%%%%%%%%%%%%%%%%%%%%%
% Twenty One Seconds Resume/CV
% LaTeX Template
% Version 1.0 (2022/12/29)
%
% This template has been downloaded from:
% http://www.LaTeXTemplates.com
%
%
% profile image is from https://publicdomainvectors.org/photos/nayrhcrel-alice-32.png
%
% License
% Original author:
% Carmine Spagnuolo (cspagnuolo@unisa.it) with major modifications by
% Alessandro Trinca Tornidor (alessandro at trinca dot tornidor dot com)
%% Copyright 2022-now Alessandro Trinca Tornidor (alessandro at trinca dot tornidor dot com)
%
% This work may be distributed and/or modified under the
% conditions of the LaTeX Project Public License, either version 1.3
% of this license or (at your option) any later version.
% The latest version of this license is in
%   http://www.latex-project.org/lppl.txt
% and version 1.3 or later is part of all distributions of LaTeX
% version 2005/12/01 or later.
%
% This work has the LPPL maintenance status `maintained'.
%
% The Current Maintainer of this work is Alessandro Trinca Tornidor
%
% This work consists of the files template.tex and twentyonesecondcv.cls
% and the derived file twentyonesecondcv.pdf

%----------------------------------------------------------------------------------------
%	PACKAGES AND OTHER DOCUMENT CONFIGURATIONS
%----------------------------------------------------------------------------------------

\documentclass[letterpaper]{twentyonesecondcv} % a4paper for A4

\usepackage{comment}

\profilepic{me.jpg} % Profile picture
\cvjobtitle{Internship, data science} % Job title/career
\cvname{Luiz G.~Mugnaini A.} % Your name
\cvaddress{São Paulo, Brasil} % Short address/location, use \newline if more than 1 line is required
\cvsitepersonal{luizmugnaini.github.io} %personal site
\cvstackoverflow{} % Personal website
\cvlinkedin{luiz-mugnaini-7a838a231}
\cvskypeurl{} % Skype
\cvgithub{luizmugnaini}
\cvmail{luizmugnaini@gmail.com} % Email address

\skills{{Research/6},{Mathematics/6},{English/6},{GNU\slash Linux/5},{Rust/4},{Python/5}}
\skillstext{}

\begin{document}
\sidesection{
    \makeheaderprofile
    \makeinfoprofile
    \makeskillsprofile
    \par\vfill
    \makefooterprofile
} % for some reason yoy can't have a new line here...
\mainsection{
%%%%%%%%%%%%%%%%%%%%%%%%%%%%%%%%%%%%%%%%%%%%%%%%%%%%%%%%%%%%%%
%%%%%%Skill bar section, each skill must have a value between 0 an 6 (float)%%%%%%%
%%%%%%%%%%%%%%%%%%%%%%%%%%%%%%%%%%%%%%%%%%%%%%%%%%%%%%%%%%%%%%
\section{About me}

I'm an undergraduate student in \href{https://www.cecm.usp.br/}{Molecular
  Sciences} at the University of Sao Paulo specialising in mathematics and
computer science. Together with my adviser
\href{https://www.ime.usp.br/~ivanstru/}{Ivan Struchiner} we are currently
studying abstract homotopy theory via model categories.

\section{Interests}

As a scientist I'm eager to build bridges between the abstract realm of
\emph{mathematics} and \emph{computer science}, more specifically in the field
of \emph{machine learning} and \emph{data science} at large. Aside from
academic research, I'm looking forward for an \emph{internship} where I would be
able to \emph{learn} how to \emph{apply} these techniques at the \emph{industry} level.

\section{Education}

\begin{twenty}
  \twentyitem
    {2020-2024}
    {B.Sc. Molecular Sciences}
    {University of Sao Paulo}
    {Molecular Sciences is a bachelor degree at the University of Sao Paulo for
    specially selected students focusing on the interplay of computer science,
    mathematics, physics, biology and chemistry. The structure of the course is
    build so that the students have the freedom of research by combining
    multiple topics of interest. My weighted average grade is 9.2/10.}
\end{twenty}

\section{Projects}

\begin{twentymid} % Environment for a list with descriptions
    \twentymiditem
    {Since 2022}{
    \href{https://link.springer.com/book/10.1007/978-3-031-10447-3}
    {\textcolor{mainblue}{Simplicial \& Dendroidal Homotopy Theory}}

    My \emph{undergraduate research} project focuses on the study of
    \href{https://ncatlab.org/nlab/show/Introduction+to+Homotopy+Theory}{\emph{homotopy
      theory}} via the lenses of simplicial sets, dendroidal sets and model
    categories. Most of my progress can be found in my publicly available
    \href{https://luizmugnaini.github.io/deep-dive}{research notes}.

    }{University of Sao Paulo - IME}

    \twentymiditem
    {~~~~~~2022}{
      \href{https://fismat.pablopie.xyz/pdf/monografias/luiz.pdf}
      {\textcolor{mainblue}{2D Topological Quantum Field Theory \& Frobenius Algebras}}

      As a final project for my
      \href{https://sites.google.com/view/cristian-ortiz/usp2022-math-physics}{\emph{mathematical-physics
          class}}, I decided to study the categorical equivalence between
      2-dimensional topological quantum field theories and the category of
      commutative Frobenius algebras. The paper can be found
      \href{https://fismat.pablopie.xyz/pdf/monografias/luiz.pdf}{here}.

    }{University of Sao Paulo}

    \twentymiditem
    {~~~~~~2022}{
      \href{https://github.com/luizmugnaini/numerical}
      {\textcolor{mainblue}{Numerical Methods}}

      As a final project for my \emph{numerical methods} class, I've developed
      together with two colleages an \emph{open source Python package}
      \href{https://github.com/luizmugnaini/numerical}{\texttt{numerical}}.

    }{University of Sao Paulo}

    \twentymiditem
    {~~~~~~2022}{
      \href{https://github.com/luizmugnaini/chirp}
      {\textcolor{mainblue}{CHIP-8 interpreter}}

      As a personal project aiming to know the ways of \emph{hardware emulation}, I
      developed a \href{https://en.wikipedia.org/wiki/CHIP-8}{CHIP-8}
      interpreter in \emph{Rust}.

    }{}

    \twentymiditem
    {~~~~~~2022}{
      \href{https://github.com/luizmugnaini/radiant}
      {\textcolor{mainblue}{Ray tracing}}

      Wishing to understand more about \emph{computer graphics} and rendering
      techniques, I implemented a
      \href{https://en.wikipedia.org/wiki/Ray_tracing_(graphics)}{ray tracer} in
      Rust.

    }{}

\end{twentymid}

% end of page
}
% \newpage

\begin{comment}
% \noindent
\sidesection{
    \makeheaderprofile
    \makeskillsprofile
    \par\vfill
    \makefooterprofile
}
\mainsection{

\section{experience}

\begin{twenty}
  \twentyitem
    {1900}
    {Alice in Wonderland-The Circra (1900's) Silent Film.}
    {Film}
    {The first Alice on film was over a hundred years ago. Lorem ipsum dolor sit amet, consectetur adipiscing elit. Nulla suscipit a nisi nec fermentum. Praesent sed leo fringilla, eleifend diam vel, ultricies mauris. Nunc interdum diam vel magna egestas posuere. Mauris sodales urna vitae neque imperdiet tempus. Nunc rhoncus a nunc ac varius. Integer condimentum sit amet dui sed fringilla. Lorem ipsum dolor sit amet, consectetur adipiscing elit. Nulla suscipit a nisi nec fermentum. Praesent sed leo fringilla, eleifend diam vel, ultricies mauris. Nunc interdum diam vel magna egestas posuere. Mauris sodales urna vitae neque imperdiet tempus. Nunc rhoncus a nunc ac varius. Integer condimentum sit amet dui sed fringilla.}
  \twentyitem
    {1933}
    {Alice in Wonderland 1933 version.}
    {Film}
    {This film stars Ethel griffies and Charlotte Henry. It was a box office flop when it was released.}
\twentyitem
    {1951}
    {Disney Film.}
    {Film}
    {Walt Disney brings Lewis Carroll's fantasy story to life in this well done animated classic. Even though many elements from the book were dropped, such as the duchess with the baby pig and mock turtle, this version is without a doubt the most famous Alice adaption made.}
\end{twenty}

\section{other information}

Alice approaches Wonderland as an anthropologist, but maintains a strong sense of noblesse oblige that comes with her class status. She has confidence in her social position, education, and the Victorian virtue of good manners. Alice has a feeling of entitlement, particularly when comparing herself to Mabel, whom she declares has a “poky little house,” and no toys. Additionally, she flaunts her limited information base with anyone who will listen and becomes increasingly obsessed with the importance of good manners as she deals with the rude creatures of Wonderland. Alice maintains a superior attitude and behaves with solicitous indulgence toward those she believes are less privileged.

Lorem ipsum dolor sit amet, consectetur adipiscing elit. Donec euismod id nisi nec dapibus. Nunc sed cursus nisi, in feugiat elit. Orci varius natoque penatibus et magnis dis parturient montes, nascetur ridiculus mus. Praesent dictum vehicula est non dapibus. Praesent id tincidunt quam. Morbi pharetra, lorem non faucibus vestibulum, turpis eros blandit felis, at mollis diam metus in elit. In venenatis pharetra leo vel elementum. Lorem ipsum dolor sit amet, consectetur adipiscing elit. Ut non nulla pellentesque, vestibulum lorem in, vehicula velit. Integer vitae fringilla ipsum. Donec vel pretium libero.

In sed ligula a turpis tristique interdum a ut risus. Nullam at euismod leo, ac molestie urna. Curabitur enim felis, ultricies quis dui ac, bibendum auctor dui. Proin aliquet nulla arcu, at pulvinar eros eleifend at. Aliquam erat volutpat. Duis augue mauris, aliquam sit amet lacinia ut, egestas sit amet est. Aliquam erat lacus, scelerisque non laoreet eu, commodo eget turpis. Nunc nunc mauris, lacinia sed semper quis, venenatis eget enim. Proin viverra in nibh at tincidunt.

Integer molestie, lectus in vehicula consequat, mauris nisi dignissim lacus, et vestibulum ante nisl eu ante. In laoreet eros sit amet justo tempor maximus. Cras quam velit, feugiat vel risus vitae, tincidunt elementum.
}
\end{comment}
\end{document}
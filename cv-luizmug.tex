%%%%%%%%%%%%%%%%%%%%%%%%%%%%%%%%%%%%%%%%%
% Twenty One Seconds Resume/CV
% LaTeX Template
% Version 1.0 (2022/12/29)
%
% This template has been downloaded from:
% http://www.LaTeXTemplates.com
%
%
% profile image is from https://publicdomainvectors.org/photos/nayrhcrel-alice-32.png
%
% License
% Original author:
% Carmine Spagnuolo (cspagnuolo@unisa.it) with major modifications by
% Alessandro Trinca Tornidor (alessandro at trinca dot tornidor dot com)
%% Copyright 2022-now Alessandro Trinca Tornidor (alessandro at trinca dot tornidor dot com)
%
% This work may be distributed and/or modified under the
% conditions of the LaTeX Project Public License, either version 1.3
% of this license or (at your option) any later version.
% The latest version of this license is in
%   http://www.latex-project.org/lppl.txt
% and version 1.3 or later is part of all distributions of LaTeX
% version 2005/12/01 or later.
%
% This work has the LPPL maintenance status `maintained'.
%
% The Current Maintainer of this work is Alessandro Trinca Tornidor
%
% This work consists of the files template.tex and twentyonesecondcv.cls
% and the derived file twentyonesecondcv.pdf

%----------------------------------------------------------------------------------------
%	PACKAGES AND OTHER DOCUMENT CONFIGURATIONS
%----------------------------------------------------------------------------------------

\documentclass[letterpaper]{twentyonesecondcv} % a4paper for A4

\usepackage{comment}

\profilepic{me.jpg} % Profile picture
\cvjobtitle{Internship, data science} % Job title/career
\cvname{Luiz G.~Mugnaini A.} % Your name
\cvaddress{São Paulo, Brasil} % Short address/location, use \newline if more than 1 line is required
\cvsitepersonal{luizmugnaini.github.io} %personal site
\cvstackoverflow{} % Personal website
\cvlinkedin{luiz-mugnaini-7a838a231}
\cvskypeurl{} % Skype
\cvgithub{luizmugnaini}
\cvmail{luizmugnaini@gmail.com} % Email address

\skills{{Research/6},{Mathematics/6},{English/6},{GNU\slash Linux/5},{Rust/4},{Python/5}}
\skillstext{}

\begin{document}
\sidesection{
    \makeheaderprofile
    \makeinfoprofile
    \makeskillsprofile
    \par\vfill
    \makefooterprofile
} % for some reason yoy can't have a new line here...
\mainsection{
%%%%%%%%%%%%%%%%%%%%%%%%%%%%%%%%%%%%%%%%%%%%%%%%%%%%%%%%%%%%%%
%%%%%%Skill bar section, each skill must have a value between 0 an 6 (float)%%%%%%%
%%%%%%%%%%%%%%%%%%%%%%%%%%%%%%%%%%%%%%%%%%%%%%%%%%%%%%%%%%%%%%
\section{About me}

I'm an undergraduate student in \href{https://www.cecm.usp.br/}{Molecular
  Sciences} at the University of Sao Paulo specialising in mathematics and
computer science. Together with my adviser
\href{https://www.ime.usp.br/~ivanstru/}{Ivan Struchiner} we are currently
studying abstract homotopy theory via model categories.

\section{Interests}

As a scientist I'm eager to build bridges between the abstract realm of
\emph{mathematics} and \emph{computer science}, more specifically in the field
of \emph{machine learning} and \emph{data science} at large. Aside from
academic research, I'm looking forward for an \emph{internship} where I would be
able to \emph{learn} how to \emph{apply} these techniques at the \emph{industry} level.

\section{Education}

\begin{twenty}
  \twentyitem
    {2020-2024}
    {B.Sc. Molecular Sciences}
    {University of Sao Paulo}
    {Molecular Sciences is a bachelor degree at the University of Sao Paulo for
    specially selected students focusing on the interplay of computer science,
    mathematics, physics, biology and chemistry. The structure of the course is
    build so that the students have the freedom of research by combining
    multiple topics of interest. My weighted average grade is 9.2/10.}
\end{twenty}

\section{Projects}

\begin{twentymid} % Environment for a list with descriptions
    \twentymiditem
    {Since 2022}{
    \href{https://link.springer.com/book/10.1007/978-3-031-10447-3}
    {\textcolor{mainblue}{Simplicial \& Dendroidal Homotopy Theory}}

    My \emph{undergraduate research} project focuses on the study of
    \href{https://ncatlab.org/nlab/show/Introduction+to+Homotopy+Theory}{\emph{homotopy
      theory}} via the lenses of simplicial sets, dendroidal sets and model
    categories. Most of my progress can be found in my publicly available
    \href{https://luizmugnaini.github.io/deep-dive}{research notes}.

    }{University of Sao Paulo - IME}

    \twentymiditem
    {~~~~~~2022}{
      \href{https://luizmugnaini.github.io/2d-tqft-frobenius}
      {\textcolor{mainblue}{2D Topological Quantum Field Theory \& Frobenius Algebras}}

      As a final project for my
      \href{https://sites.google.com/view/cristian-ortiz/usp2022-math-physics}{\emph{mathematical-physics
          class}}, I decided to study the categorical equivalence between
      2-dimensional topological quantum field theories and the category of
      commutative Frobenius algebras. The paper can be found
      \href{https://luizmugnaini.github.io/2d-tqft-frobenius}{here}.

    }{University of Sao Paulo}

    \twentymiditem
    {~~~~~~2022}{
      \href{https://github.com/luizmugnaini/numerical}
      {\textcolor{mainblue}{Numerical Methods}}

      As a final project for my \emph{numerical methods} class, I've developed
      together with two colleages an \emph{open source Python package}
      \href{https://github.com/luizmugnaini/numerical}{\texttt{numerical}}.

    }{University of Sao Paulo}

    \twentymiditem
    {~~~~~~2022}{
      \href{https://github.com/luizmugnaini/chirp}
      {\textcolor{mainblue}{CHIP-8 interpreter}}

      As a personal project aiming to know the ways of \emph{hardware
        emulation}, I developed
      a \href{https://en.wikipedia.org/wiki/CHIP-8}{CHIP-8} interpreter in
      \emph{Rust}---the project can be
      found \href{https://github.com/luizmugnaini/chirp}{here}.

    }{}

    \twentymiditem
    {~~~~~~2022}{
      \href{https://github.com/luizmugnaini/radiant}
      {\textcolor{mainblue}{Ray tracing}}

      Wishing to understand more about \emph{computer graphics} and rendering
      techniques, I implemented
      a \href{https://en.wikipedia.org/wiki/Ray_tracing_(graphics)}{ray tracer}
      in \emph{Rust}---the project can be
      found \href{https://github.com/luizmugnaini/radiant}{here}.

    }{}

\end{twentymid}

% end of page
}

\end{document}